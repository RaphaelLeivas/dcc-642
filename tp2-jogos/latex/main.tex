%File: formatting-instructions-latex-2024.tex
%release 2024.0
\documentclass[letterpaper]{article} % DO NOT CHANGE THIS
\usepackage{aaai24}  % DO NOT CHANGE THIS
\usepackage{times}  % DO NOT CHANGE THIS
\usepackage{helvet}  % DO NOT CHANGE THIS
\usepackage{courier}  % DO NOT CHANGE THIS
\usepackage[hyphens]{url}  % DO NOT CHANGE THIS
\usepackage{graphicx} % DO NOT CHANGE THIS
\usepackage{indentfirst}
\usepackage[portuguese]{babel}
\urlstyle{rm} % DO NOT CHANGE THIS
\def\UrlFont{\rm}  % DO NOT CHANGE THIS
\usepackage{natbib}  % DO NOT CHANGE THIS AND DO NOT ADD ANY OPTIONS TO IT
\usepackage{caption} % DO NOT CHANGE THIS AND DO NOT ADD ANY OPTIONS TO IT
\frenchspacing  % DO NOT CHANGE THIS
\setlength{\pdfpagewidth}{8.5in}  % DO NOT CHANGE THIS
\setlength{\pdfpageheight}{11in}  % DO NOT CHANGE THIS
%
% These are recommended to typeset algorithms but not required. See the subsubsection on algorithms. Remove them if you don't have algorithms in your paper.
\usepackage{algorithm}
% \usepackage{algorithmic}
\usepackage{algpseudocode}

%
% These are are recommended to typeset listings but not required. See the subsubsection on listing. Remove this block if you don't have listings in your paper.
\usepackage{newfloat}
\usepackage{listings}
\DeclareCaptionStyle{ruled}{labelfont=normalfont,labelsep=colon,strut=off} % DO NOT CHANGE THIS
\lstset{%
	basicstyle={\footnotesize\ttfamily},% footnotesize acceptable for monospace
	numbers=left,numberstyle=\footnotesize,xleftmargin=2em,% show line numbers, remove this entire line if you don't want the numbers.
	aboveskip=0pt,belowskip=0pt,%
	showstringspaces=false,tabsize=2,breaklines=true}
\floatstyle{ruled}
\newfloat{listing}{tb}{lst}{}
\floatname{listing}{Listing}
%
% Keep the \pdfinfo as shown here. There's no need
% for you to add the /Title and /Author tags.
\pdfinfo{
/TemplateVersion (2024.1)
}

% DISALLOWED PACKAGES
% \usepackage{authblk} -- This package is specifically forbidden
% \usepackage{balance} -- This package is specifically forbidden
% \usepackage{color (if used in text)
% \usepackage{CJK} -- This package is specifically forbidden
% \usepackage{float} -- This package is specifically forbidden
% \usepackage{flushend} -- This package is specifically forbidden
% \usepackage{fontenc} -- This package is specifically forbidden
% \usepackage{fullpage} -- This package is specifically forbidden
% \usepackage{geometry} -- This package is specifically forbidden
% \usepackage{grffile} -- This package is specifically forbidden
% \usepackage{hyperref} -- This package is specifically forbidden
% \usepackage{navigator} -- This package is specifically forbidden
% (or any other package that embeds links such as navigator or hyperref)
% \indentfirst} -- This package is specifically forbidden
% \layout} -- This package is specifically forbidden
% \multicol} -- This package is specifically forbidden
% \nameref} -- This package is specifically forbidden
% \usepackage{savetrees} -- This package is specifically forbidden
% \usepackage{setspace} -- This package is specifically forbidden
% \usepackage{stfloats} -- This package is specifically forbidden
% \usepackage{tabu} -- This package is specifically forbidden
% \usepackage{titlesec} -- This package is specifically forbidden
% \usepackage{tocbibind} -- This package is specifically forbidden
% \usepackage{ulem} -- This package is specifically forbidden
% \usepackage{wrapfig} -- This package is specifically forbidden
% DISALLOWED COMMANDS
% \nocopyright -- Your paper will not be published if you use this command
% \addtolength -- This command may not be used
% \balance -- This command may not be used
% \baselinestretch -- Your paper will not be published if you use this command
% \clearpage -- No page breaks of any kind may be used for the final version of your paper
% \columnsep -- This command may not be used
% \newpage -- No page breaks of any kind may be used for the final version of your paper
% \pagebreak -- No page breaks of any kind may be used for the final version of your paperr
% \pagestyle -- This command may not be used
% \tiny -- This is not an acceptable font size.
% \vspace{- -- No negative value may be used in proximity of a caption, figure, table, section, subsection, subsubsection, or reference
% \vskip{- -- No negative value may be used to alter spacing above or below a caption, figure, table, section, subsection, subsubsection, or reference

\setcounter{secnumdepth}{2} %May be changed to 1 or 2 if section numbers are desired.

% The file aaai24.sty is the style file for AAAI Press
% proceedings, working notes, and technical reports.
%

% Title


\title{Universidade Federal de Minas Gerais \\[10pt]
DCC642 - Introdução à Inteligência Artificial (2025/2) \\
 TP2: Busca Competitiva}
\author {
    Raphael Henrique Braga Leivas - 2020028101
}


\begin{document}

\maketitle


\section{Introdução}

Neste trabalho, algoritmos de busca competitiva são implementados em Python para 
competirem em um jogo de Ligue-4. Diferentes algoritmos são implementados e sua performance 
é comparada experimentalmente com base em diferentes métricas. 

Os algoritmos Minimax, Minimax com poda Alfa-Beta e Iterative Deepening 
são algoritmos usadas em jogos de dois jogadores para escolher o 
melhor lance possível. 
O Minimax simula jogadas futuras alternando turnos de maximizar e 
minimizar para avaliar estados do jogo e selecionar o que leva ao 
melhor resultado garantido, usando uma heurística de avaliação do estado 
atual do tabuleiro. 

A poda Alfa-Beta é uma otimização do Minimax que reduz o número de nós 
analisados ao descartar ramos da árvore de decisão que não podem 
influenciar o resultado final, mantendo a mesma qualidade da decisão, 
porém com maior eficiência. 

Já o Iterative Deepening combina a precisão do Minimax 
com o limite de tempo, realizando buscas sucessivas com profundidades 
crescentes e aproveitando resultados anteriores para melhorar a
ordem de exploração dos nós, permitindo encontrar a melhor jogada 
possível dentro do tempo disponível.

As implementações dos algoritmos nesse trabalho são baseadas nas implementações 
de \cite{russell2020aima}.


\section{Objetivos}

Os objetivos principais do trabalho são: 

\begin{itemize}
	\item Implementar em Python os algoritmos Minimax, Minimax com poda Alfa-Beta, 
	Iterative Deepening com uma função heurística de avaliação do estado atual do tabuleiro;
	\item Comparar as performances dos algoritmos com base nas seguintes métricas:  
	taxa de vitória, tempo médio por jogada, média de estados visitados 
\end{itemize}

\section{Metodologia}

\subsection{Heurística de Avaliação}

O primeiro passo é definir a heurística de avaliação. Definimos uma heurística da seguinte 
forma:

\begin{itemize}
	\item Para cada sequência de 4 posições na horizontal, vertical ou diagonal, adicionamos um valor 
	em uma ordem de grandeza: sequência de duas peças soma-se 10, sequência de três peças soma-se 100, quatro, 1000. 
	\item O valor atribuído varia no sentido positivo indicando que o estado é favorável ao jogar 2, e no sentido negativo 
	para o jogador 1. Assim, o jogador P1 assume o papel de minimizar e o P2 de maximizar.
	\item As células do centro são mais importantes no jogo, e portanto são multiplicadas por um fator 6.
\end{itemize}

Dessa forma, a heurística avalia os tabuleiros como mostra a Figura \ref{fig:heuristica}.

\begin{figure}[htb]
	\centering 
    \caption{Exemplos de avaliação do estado atual do jogo com a heurística definida.}
	\includegraphics[width=\columnwidth]{images/heuristica.png}
	\label{fig:heuristica}
\end{figure}

Na Figura \ref{fig:heuristica} (a), o jogo parece bem empatado, com ambos jogadores sem ameaças iminentes e com controle
central disputado. Assim, a avaliação é próxima de zero: $f(n) = 16$. Em (b), o jogador amarelo tem sequências de 3 peças 
sem bloqueio e bom controle central, de modo que $f(n) = 222$, um valor positivo elevado. Em (c) temos o contrário de (b), 
logo a heurística é um valor elevado negativo: $f(n) = - 308$. Por fim, no caso de vitória de um jogador como ocorre em (d), 
a função retorna $\pm \infty$.

\subsection{Agente Minimax}

O primeiro agente implementado é o Minimax. O jogador vermelho assume o papel de  
minimizar e o amarelo de maximizar. O Algoritmo \ref{alg:minimax} mostra 
o pseudocódigo para o Minimax usado no projeto. Os experimentos comparam a performance 
do minimax para diferentes valores de profundidade $depth$, bem como o tempo de execução.

\begin{algorithm}[H]
\caption{Minimax.} \label{alg:minimax}
\begin{algorithmic}[1]
\Function{Minimax}{$state, depth, maxPlayer$}
    \If{$depth = 0$ or \Call{IsTerminal}{$state$}}
        \State \Return \Call{Evaluate}{$state$}
    \EndIf

    \If{$maxPlayer$}
        \State $maxEval \gets -\infty$
        \ForAll{$child$ in \Call{Successors}{$state$}}
            \State $eval \gets$ \Call{Minimax}{$child, depth - 1, \textbf{false}$}
            \State $maxEval \gets \max(maxEval, eval)$
        \EndFor
        \State \Return $maxEval$
    \Else
        \State $minEval \gets +\infty$
        \ForAll{$child$ in \Call{Successors}{$state$}}
            \State $eval \gets$ \Call{Minimax}{$child, depth - 1, \textbf{true}$}
            \State $minEval \gets \min(minEval, eval)$
        \EndFor
        \State \Return $minEval$
    \EndIf
\EndFunction
\end{algorithmic}
\end{algorithm}

\subsection{Agente Minimax com Poda Alfa Beta}

Para adicionar a Poda Alfa Beta no Algoritmo \ref{alg:minimax}, basta adicionar a seguinte 
condicional no algoritmo dentro do loop dos sucessores:

\begin{algorithm}[H]
\caption{Minimax com Poda Alfa Beta.} \label{alg:alfa-beta}
\begin{algorithmic}[1]
\If{$\alpha \geq \beta$}
	\State \textbf{break} 
\EndIf
\end{algorithmic}
\end{algorithm}

Note que $\alpha$ e $\beta$ agora são argumentos passados para a função Minimax.
O número de nós expandidos com a poda é comparado com o Minimax sem poda, de modo a verificar experimentalmente 
o impacto da poda na execução do algoritmo.

\subsection{Iterative Deepening}

Para implementar o Iterative Deepening, basta chamar o Minimax várias vezes
incrementando a profundidade máxima até estourar o limite de tempo 
estipulado, como mostra o Algoritmo \ref{alg:ids}.
Usamos a poda Alfa-Beta dado que o Minimax é chamado várias vezes, de modo a 
maximizar o número de chamadas dentro do limite de tempo estipulado.

\begin{algorithm}[H]
\caption{Iterative Deepening com Minimax} \label{alg:ids}
\begin{algorithmic}[1]
\Function{IterativeDeepening}{$state, player$}
    \While{$time < MAX\_TIME$}
        \State $depth \gets depth + 1$
        \State $bestMove \gets$ \Call{Minimax}{$child, depth, player$}
    \EndWhile

    \State \Return $bestMove$
\EndFunction
\end{algorithmic}
\end{algorithm}

Quando o limite de tempo é atingido no Algoritmo \ref{alg:ids}, ele retorna 
o último melhor lance obtido na profundidade anterior.
Note que o servidor Python encerra o processo quando o tempo limite é atingido, 
usando um lance aleatório de fallback nesse caso. Para evitar isso, usamos um 
tolerância de 250 ms para que o Iterative Deepening se encerre completamente antes 
do processo ser finalizado pelo servidor.

\subsection{Experimentos}

Os seguintes experimentos serão realizados:

\begin{itemize}
	\item Minimax vs Aleatório 
	\item Alfa-Beta vs Minimax (sem poda)
	\item Iterative Deepening vs Alfa-Beta
	\item IA do Aluno vs Jogador Humano
\end{itemize}

Os experimentos são realizados com o seguinte procedimento:

\begin{enumerate}
	\item Realiza 3 jogos entre o Minimax e o oponente para cada nível de profundidade; 
	\item Salva os tempos por jogada e número de nós expandidos em cada jogada em um csv;
	\item Analisa os dados a posteriori com \verb|matplotlib| e extrai as conclusões.
\end{enumerate}

\section{Resultados}

\subsection{Minimax vs Aleatório}

A Figura \ref{fig:histograma-minimax-time} mostra os tempos por lance para diferentes profundidades do Minimax, e a
Figura \ref{fig:histograma-minimax-nodes} exibe o número de nós expandidos
com diferentes profundidades configuradas. Como esperado, o algoritmo gasta mais tempo e expande mais nós para 
profundidades maiores.

\begin{figure}[htb]
	\centering 
    \caption{Histograma de tempos gasto por lance para o Minimax para diferentes profundidades.}
	\includegraphics[width=\columnwidth]{images/histograma-minimax-time.png}
	\label{fig:histograma-minimax-time}
\end{figure}

\begin{figure}[htb]
	\centering 
    \caption{Histograma de nós expandidos por lance para o Minimax para diferentes profundidades.}
	\includegraphics[width=\columnwidth]{images/histograma-minimax-nodes.png}
	\label{fig:histograma-minimax-nodes}
\end{figure}

Podemos tomar a média e o desvio padrão para cada uma das medições acima, e junto 
com a taxa de vitórias para cada profundidade, obtemos os resultados da Tabela \ref{tab:minimax-aleatorio}.

\begin{table}[htbp] 
    \centering 
    \caption{Resultados finais do experimento Minimax vs Aleatório.} 
    \label{tab:minimax-aleatorio} 

    \begin{tabular}{c|c|c|c} 
        Depth & Vitórias (\%) & Tempo (ms) & Nós \\
		\hline 
        2 & 0 & $3.6 \pm 4.9$ & $48.2 \pm 12$\\
        3 & 100 & $14.1 \pm 7.4$ & $384 \pm 30$\\
        4 & 100 & $92.5 \pm 33.2$ & $2357 \pm 695$\\
        5 & 100 & $618 \pm 130$ & $16769 \pm 2354$\\
    \end{tabular}
\end{table}

\subsection{Alfa-beta vs Minimax}

Repetindo o mesmo procedimento do primeiro experimento, mas dessa 
vez usando o minimax com a poda alta-beta vs o minimax, obtemos os 
resultados exibidos na Tabela \ref{tab:minimax-alfabeta}

\begin{table}[htbp] 
    \centering 
    \caption{Resultados finais do experimento Minimax Alfa-Beta vs Minimax.} 
    \label{tab:minimax-alfabeta} 

    \begin{tabular}{c|c|c|c} 
        Depth & Vitórias (\%) & Tempo (ms) & Nós \\
		\hline 
        2 & 0 & $1.7 \pm 2.7$ & $31 \pm 7.96$\\
        3 & 100 & $6.4 \pm 3.8$ & $125.6 \pm 52.2$\\
        4 & 0 & $10.1 \pm 5.9$ & $265 \pm 107.7$\\
        5 & 100 & $83.2 \pm 66.3$ & $2366.4 \pm 1916$\\
    \end{tabular}
\end{table}

Comparando as Tabelas \ref{tab:minimax-aleatorio} e \ref{tab:minimax-alfabeta},
vemos que a poda causou uma redução significativa nos tempos e números de nós  
expandidos a cada lance, em particular para maiores profundidades.
A Figura \ref{fig:diff-alfabeta} sumariza as diferenças entre os tempos e 
números de nós visitados para ambos algoritmos, destacando a redução com 
a poda alfa-beta.

\begin{figure}[htb]
	\centering 
    \caption{Diferenças entre os tempos e números de nós expandidos pela profundidade entre os algoritmos.}
	\includegraphics[width=\columnwidth]{images/diff-alfabeta-time.png}
	\includegraphics[width=\columnwidth]{images/diff-alfabeta-nodes.png}
	\label{fig:diff-alfabeta}
\end{figure}

\subsection{Iterative Deepening vs Alfa-beta}

Nesse experimento, o Minimax com poda Alfa-beta foi fixado com profundidade igual a 5, 
enquanto o IDS pode ir até a profundidade desejada dentro do limite de tempo 
estipulado (menos a tolerância de 250 ms).

A Tabela \ref{tab:ids-alphabeta} sumariza os resultados obtidos 
nesse experimento. O IDS obteve uma taxa de vitória de 66\% e 100\% para 
os limites de tempo de 1000 e 2000 ms, respectivamente. 
O número de nós visitados pelo IDS é significativamente maior que o Alfa Beta 
da Tabela \ref{tab:minimax-alfabeta}. 

\begin{table}[htbp] 
    \centering 
    \caption{Resultados finais do experimento Iterative Deepning vs Alfa-beta.} 
    \label{tab:ids-alphabeta} 

    \begin{tabular}{c|c|c|c|c} 
        Lim & V & Tempo & Nós & Depth \\
        (ms) & (\%) & (ms) & & (mediana) \\
		\hline 
        1000 & 66 & $752.4 \pm 2.56$ & $6891.5 \pm 643$ & 7 \\
        2000 & 100 & $1751.3 \pm 1.46$ & $27017 \pm 7251$ & 7 \\
    \end{tabular}
\end{table}

A mediana da profundidade atingida pelo IDS foi de 7 em ambos os limites de tempo. 
A Figura \ref{fig:tempos-ids} mostra as distribuições de tempos gasto por lance para cada um dos limites 
com o IDS. Note que os tempos possuem uma variância muito menor que os outros casos uma vez 
que o próprio código finalizava a execução quando chegava em 750 e 1750 ms respectivamente, 
retornando o lance encontrado na última profundidade até esse instante.

\begin{figure}[htb]
	\centering 
    \caption{Tempos por lance para cada uma das profundidades com o Iterative Search.}
	\includegraphics[width=\columnwidth]{images/tempos-ids.png}
	\label{fig:tempos-ids}
\end{figure}

\subsection{Humano vs Alfa-beta}

A melhor IA obtida foi o Minimax com poda Alfa-beta e profundidade 5. 
Ao jogar 5 partidas contra ela, obtive os resultados exibidos na 
Tabela \ref{tab:humano-ia}.

\begin{table}[htbp] 
    \centering 
    \caption{Resultados finais das partidas humano versus melhor IA.} 
    \label{tab:humano-ia} 

    \begin{tabular}{c|c|c} 
        Partida & Vencedor & Número de Lances \\
		\hline 
        1 & IA & $18$ \\
        2 & IA & $14$ \\
        3 & IA & $12$ \\
        4 & Empate & $22$ \\
        5 & IA & $17$ \\
    \end{tabular}
\end{table}

Não consegui vencer a melhor IA em nenhuma das partidas. Quase sempre eu perdia 
por não ver alguma diagonal ameaçada. Uma outra observação interessante é que a IA sempre 
começa com a mesma jogada, exibida na Figura \ref{fig:jogada-inicial-ia}.
Se eu não bloquear a primeira linha, ela sempre irá colocar nas colunas adjacentes 
e ameaçar colocar 3 bolas em sequência com os extremos abertos, uma ameaça 
dupla que vence o jogo.

\begin{figure}[htb]
	\centering 
    \caption{Jogada inicial da IA em todas as partidas.}
	\includegraphics[width=0.8\columnwidth]{images/jogada-inicial-ia.png}
	\label{fig:jogada-inicial-ia}
\end{figure}

A jogada inicial da Figura \ref{fig:jogada-inicial-ia} só ocorre em profundidade maior que 4.
Para profundidades 2 e 3, ela começa em um dos cantos do tabuleiro.


\section{Conclusão}

Tendo em vista os objetivos do trabalho, foi possível comparar experimentalmente 
a performance de três algoritmos de busca competitiva vistos em sala de aula 
no jogo de Ligue-4, bem como os efeitos que a função heurística tem na 
execução do algoritmo Minimax. 

Além disso, foi possível verificar como a poda Alfa-beta consegue reduzir substancialmente 
o tempo gasto por lance e o número de nós expandidos. O Iterative Deepening também 
demonstra um grande potencial quando aliado ao Alfa-beta, dado que ele consegue atingir profundidades 
maiores dentro do limite de tempo estipulado com o ganho de eficiência proporcionado pela 
poda Alfa-beta.

\bibliography{aaai24}

\end{document}
